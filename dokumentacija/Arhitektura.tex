\chapter{Arhitektura i dizajn sustava}
		
		\textbf{\textit{dio 1. revizije}}\\

		\textit{ Potrebno je opisati stil arhitekture te identificirati: podsustave, preslikavanje na radnu platformu, spremišta podataka, mrežne protokole, globalni upravljački tok i sklopovsko-programske zahtjeve. Po točkama razraditi i popratiti odgovarajućim skicama:}
	\begin{itemize}
		\item 	\textit{izbor arhitekture temeljem principa oblikovanja pokazanih na predavanjima (objasniti zašto ste baš odabrali takvu arhitekturu)}
		\item 	\textit{organizaciju sustava s najviše razine apstrakcije (npr. klijent-poslužitelj, baza podataka, datotečni sustav, grafičko sučelje)}
		\item 	\textit{organizaciju aplikacije (npr. slojevi frontend i backend, MVC arhitektura) }		
	\end{itemize}

	
		

		

				
		\section{Baza podataka}
			
			\textbf{\textit{dio 1. revizije}}\\
			
		\textit{Potrebno je opisati koju vrstu i implementaciju baze podataka ste odabrali, glavne komponente od kojih se sastoji i slično.}
		
		Za našu web aplikaciju koristiti ćemo relacijsku bazu podataka koja je industrijski standard te najjednostavniji način za rješenje našeg problema. Osnovni element baze je relacija čija su obilježja njeno ime i atributi. Glavna zadaća naše baze je spajanje njenih korisnika i sustava s korisnikom, bilo kroz poruke ili kroz razne obrasce.
		Entitet ove baze podataka su:
		
			\begin{packed_item}
				\item Roditelj
				\item Dijete
				\item Zaposlenik
				\item Račun
				\item Poruka
				\item Bolest
				\item Bolnice
			\end{packed_item}
		
		
			\subsection{Opis tablica}
			

				\textit{Svaku tablicu je potrebno opisati po zadanom predlošku. Lijevo se nalazi točno ime varijable u bazi podataka, u sredini se nalazi tip podataka, a desno se nalazi opis varijable. Svjetlozelenom bojom označite primarni ključ. Svjetlo plavom označite strani ključ}
				
				
				\textbf{Roditelj} Ovaj entitet sadrži sve informacije o roditelju spremljenom u aplikaciji. Njegovi atributi su: OIB, ime, prezime, email njegovog poslodavca, datum rođenja, adresa stanovanja i OIB njihovog zadanog liječnika. Email poslodavca i mjesto stanovanja mogu biti prazni. Ovaj entitet je u \textit{One-to-Many} vezi s entitetom Dijete preko OIB-a,\textit{Many-to-One} vezi s doktorom preko OIB-a, \textit{One-to-One} vezi s računom preko OIB-a, \textit{Many-to-Many} vezi s Poruka preko OIB-a.
				
				\begin{longtblr}[
					label=none,
					entry=none
					]{
						width = \textwidth,
						colspec={|X[6,l]|X[6, l]|X[20, l]|}, 
						rowhead = 1,
					} %definicija širine tablice, širine stupaca, poravnanje i broja redaka naslova tablice
					\hline \SetCell[c=3]{c}{\textbf{Roditelj}}	 \\ \hline[3pt]
					\SetCell{LightGreen}OIB & INT	&  	OIB roditelja  	\\ \hline
					ime	& VARCHAR & ime roditelja   	\\ \hline 
					prezime & VARCHAR & prezime roditelja   \\ \hline 
					mail & VARCHAR	& email poslodavca roditelja (opcionalno)  \\ \hline
					datum & DATETIME & datum rođenja roditelja   \\ \hline
					mjesto & VARCHAR & adresa stanovanja (opcionalno)   \\ \hline   
					\SetCell{LightBlue} OIBzap	& INT & OIB liječnika  	\\ \hline 
				\end{longtblr}
				
				\textbf{Dijete} Ovaj entitet sadrži sve informacije o djetetu spremljenom u aplikaciji. Njegovi atributi su: OIB, ime, prezime, email njegovog poslodavca, datum rođenja, adresa stanovanja, OIB njihovog zadanog liječnika i OIB roditelja. Email vrtića i mjesto stanovanja mogu biti prazni. Ovaj entitet je u \textit{Many-to-One} vezi s entitetom Roditelj preko OIB-a,\textit{One-to-Many} vezi s doktorom preko OIB-a, \textit{Many-to-Many} vezi s Poruka preko OIB-a.
				
				\begin{longtblr}[
					label=none,
					entry=none
					]{
						width = \textwidth,
						colspec={|X[6,l]|X[6, l]|X[20, l]|}, 
						rowhead = 1,
					} %definicija širine tablice, širine stupaca, poravnanje i broja redaka naslova tablice
					\hline \SetCell[c=3]{c}{\textbf{Dijete}}	 \\ \hline[3pt]
					\SetCell{LightGreen}OIB & INT	&  	OIB djeteta  	\\ \hline
					ime	& VARCHAR & ime djeteta   	\\ \hline 
					prezime & VARCHAR & prezime djeteta   \\ \hline 
					mail & VARCHAR	& email vrtića djeteta (opcionalno)  \\ \hline
					datum & DATETIME & datum rođenja djeteta   \\ \hline
					mjesto & VARCHAR & adresa stanovanja (opcionalno)   \\ \hline   
					\SetCell{LightBlue} OIBzap	& INT & OIB pedijatra	\\ \hline
					\SetCell{LightBlue} OIBrod	& INT & OIB roditelja	\\ \hline 
				\end{longtblr}
				
				\textbf{Zaposlenik} Ovaj entitet sadrži sve informacije o zdravstvenom zaposleniku spremljenom u aplikaciji. Njegovi atributi su: OIB, ime, prezime i uloga. Uloga može biti ili liječnik ili pedijatar. Ovaj entitet je u \textit{One-to-Many} vezi s entitetom Roditelj preko OIB-a,\textit{One-to-Many} vezi s Dijete preko OIB-a,\textit{One-to-One} vezi s računom preko OIB-a, \textit{Many-to-Many} vezi s Poruka preko OIB-a.
				
				\begin{longtblr}[
					label=none,
					entry=none
					]{
						width = \textwidth,
						colspec={|X[6,l]|X[6, l]|X[20, l]|}, 
						rowhead = 1,
					} %definicija širine tablice, širine stupaca, poravnanje i broja redaka naslova tablice
					\hline \SetCell[c=3]{c}{\textbf{Zaposlenik}}	 \\ \hline[3pt]
					\SetCell{LightGreen}OIB & INT	&  	OIB zaposlenika 	\\ \hline
					ime	& VARCHAR & ime zaposlenika   	\\ \hline 
					prezime & VARCHAR & prezime zaposlenika   \\ \hline 
					uloga & VARCHAR	& profesija (liječnik ili pedijatar)  \\ \hline
				\end{longtblr}
				
				\textbf{Račun} Ovaj entitet sadrži sve informacije o računu korisnika spremljenom u aplikaciji. Njegovi atributi su: OIB, lozinka, povlaštenost. Ako je administratorksi račun onda je atribut povlaštenost jednak 1, ako ne onda je 0. Ovaj entitet je u \textit{One-to-One} vezi s entitetom Roditelj preko OIB-a,\textit{One-to-Many} vezi s Dijete preko OIB-a,\textit{One-to-One} vezi s računom preko OIB-a.
				
				\begin{longtblr}[
					label=none,
					entry=none
					]{
						width = \textwidth,
						colspec={|X[6,l]|X[6, l]|X[20, l]|}, 
						rowhead = 1,
					} %definicija širine tablice, širine stupaca, poravnanje i broja redaka naslova tablice
					\hline \SetCell[c=3]{c}{\textbf{Račun}}	 \\ \hline[3pt]
					\SetCell{LightBlue}OIB & INT	&  	OIB korisnika računa 	\\ \hline
					šifra	& VARCHAR & šifra računa  	\\ \hline 
					povlaštenost & INT & administratorska prava (0 ili 1)   \\ \hline 
				\end{longtblr}
				
				\textbf{Poruka} Ovaj entitet sadrži sve informacije o porukama spremljenima u aplikaciji. Njegovi atributi su: id, OIBpoš, OIBpri, naslov, tijelo, prilog, tip, dijagnoza. Prilog i dijagnoza mogu biti prazne, ovisno o tipu poruke. Ovaj entitet je u \textit{Many-to-Many} vezama s entitetima Roditelj i Zaposlenik preko OIB-a,\textit{One-to-One} vezi s entitetom Bolest preko id-a.
				
				\begin{longtblr}[
					label=none,
					entry=none
					]{
						width = \textwidth,
						colspec={|X[6,l]|X[6, l]|X[20, l]|}, 
						rowhead = 1,
					} %definicija širine tablice, širine stupaca, poravnanje i broja redaka naslova tablice
					\hline \SetCell[c=3]{c}{\textbf{Poruka}}	 \\ \hline[3pt]
					\SetCell{LightGreen}id & INT	&  	identifikacijski ključ poruke	\\ \hline
					\SetCell{LightBlue} OIBpoš	& INT & OIB pošiljatelja	\\ \hline
					\SetCell{LightBlue} OIBpri	& INT & OIB primatelja	\\ \hline
					naslov	& VARCHAR & naslov poruke  	\\ \hline
					tijelo	& VARCHAR & tekstualni sadržaj poruke  	\\ \hline
					prilog	& VARCHAR & link na poslanu sliku unutar datotečnog sustava aplikacije (opcionalno)	\\ \hline
					tip	& VARCHAR & tip poslane poruke (standardna,ispričnica itd.)	\\ \hline
					\SetCell{LightBlue}dijagnoza	& INT & id dijagnosticirane bolesti (opcionalno)	\\ \hline      
				\end{longtblr}
				
				\textbf{Bolest} Ovaj entitet sadrži sve informacije o bolestima spremljenima u aplikaciji. Njegovi atributi su: id i naziv. Ovaj entitet je u \textit{Many-to-Many} vezama s entitetom Bolnica id-a bolnice,\textit{One-to-One} vezi s entitetom Poruka preko id-a bolesti.
				
				\begin{longtblr}[
					label=none,
					entry=none
					]{
						width = \textwidth,
						colspec={|X[6,l]|X[6, l]|X[20, l]|}, 
						rowhead = 1,
					} %definicija širine tablice, širine stupaca, poravnanje i broja redaka naslova tablice
					\hline \SetCell[c=3]{c}{\textbf{Bolest}}	 \\ \hline[3pt]
					\SetCell{LightGreen}id & INT	&  	identifikacijski ključ bolesti	\\ \hline
					naziv	& VARCHAR & naziv bolesti	\\ \hline   
				\end{longtblr}
				
				\textbf{Bolnica} Ovaj entitet sadrži sve informacije o bolnici spremljenima u aplikaciji. Njegovi atributi su: id,naziv i adresa. Ovaj entitet je u \textit{Many-to-Many} vezama s entitetom Bolest preko id-a bolesti.
				
				\begin{longtblr}[
					label=none,
					entry=none
					]{
						width = \textwidth,
						colspec={|X[6,l]|X[6, l]|X[20, l]|}, 
						rowhead = 1,
					} %definicija širine tablice, širine stupaca, poravnanje i broja redaka naslova tablice
					\hline \SetCell[c=3]{c}{\textbf{Bolnica}	 \\ \hline[3pt]
					\SetCell{LightGreen}id & INT	&  	identifikacijski ključ bolesti	\\ \hline
					naziv	& VARCHAR & naziv bolnice	\\ \hline
					adresa	& VARCHAR & adresa bolnice	\\ \hline    
				\end{longtblr}
			
			\subsection{Dijagram baze podataka}
				\textit{ U ovom potpoglavlju potrebno je umetnuti dijagram baze podataka. Primarni i strani ključevi moraju biti označeni, a tablice povezane. Bazu podataka je potrebno normalizirati. Podsjetite se kolegija "Baze podataka".}
			
			\eject
			
			
		\section{Dijagram razreda}
		
			\textit{Potrebno je priložiti dijagram razreda s pripadajućim opisom. Zbog preglednosti je moguće dijagram razlomiti na više njih, ali moraju biti grupirani prema sličnim razinama apstrakcije i srodnim funkcionalnostima.}\\
			
			\textbf{\textit{dio 1. revizije}}\\
			
			\textit{Prilikom prve predaje projekta, potrebno je priložiti potpuno razrađen dijagram razreda vezan uz \textbf{generičku funkcionalnost} sustava. Ostale funkcionalnosti trebaju biti idejno razrađene u dijagramu sa sljedećim komponentama: nazivi razreda, nazivi metoda i vrste pristupa metodama (npr. javni, zaštićeni), nazivi atributa razreda, veze i odnosi između razreda.}\\
			
			\textbf{\textit{dio 2. revizije}}\\			
			
			\textit{Prilikom druge predaje projekta dijagram razreda i opisi moraju odgovarati stvarnom stanju implementacije}
			
			
			
			\eject
		
		\section{Dijagram stanja}
			
			
			\textbf{\textit{dio 2. revizije}}\\
			
			\textit{Potrebno je priložiti dijagram stanja i opisati ga. Dovoljan je jedan dijagram stanja koji prikazuje \textbf{značajan dio funkcionalnosti} sustava. Na primjer, stanja korisničkog sučelja i tijek korištenja neke ključne funkcionalnosti jesu značajan dio sustava, a registracija i prijava nisu. }
			
			
			\eject 
		
		\section{Dijagram aktivnosti}
			
			\textbf{\textit{dio 2. revizije}}\\
			
			 \textit{Potrebno je priložiti dijagram aktivnosti s pripadajućim opisom. Dijagram aktivnosti treba prikazivati značajan dio sustava.}
			
			\eject
		\section{Dijagram komponenti}
		
			\textbf{\textit{dio 2. revizije}}\\
		
			 \textit{Potrebno je priložiti dijagram komponenti s pripadajućim opisom. Dijagram komponenti treba prikazivati strukturu cijele aplikacije.}