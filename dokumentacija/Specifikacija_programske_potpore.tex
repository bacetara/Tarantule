\chapter{Specifikacija programske potpore}
		
	\section{Funkcionalni zahtjevi}
			
			\textbf{\textit{dio 1. revizije}}\\
			
			\textit{Navesti \textbf{dionike} koji imaju \textbf{interes u ovom sustavu} ili  \textbf{su nositelji odgovornosti}. To su prije svega korisnici, ali i administratori sustava, naručitelji, razvojni tim.}\\
				
			\textit{Navesti \textbf{aktore} koji izravno \textbf{koriste} ili \textbf{komuniciraju sa sustavom}. Oni mogu imati inicijatorsku ulogu, tj. započinju određene procese u sustavu ili samo sudioničku ulogu, tj. obavljaju određeni posao. Za svakog aktora navesti funkcionalne zahtjeve koji se na njega odnose.}\\
			
			
			\noindent \textbf{Dionici:}
			
			\begin{packed_enum}
				
				\item Roditelji
				\item Zaposlenici u zdravstvenim ustanovama
					\begin{packed_enum}
						\item Liječnici obiteljske medicine
						\item Pedijatri
					\end{packed_enum}				
				\item Administrator
				\item Razvojni tim
				
			\end{packed_enum}
			
			\noindent \textbf{Aktori i njihovi funkcionalni zahtjevi:}
			
			
			\begin{packed_enum}
				\item  \underbar{Neregistrirani/neprijavljeni korisnik (inicijator) može:}
				
				\begin{packed_enum}
					
					\item pročitati opis stranice
					\item se registrirati u sustav, za što mu je potreban OIB te lozinka
					\item se prijaviti u sustav, za što mu je potreban OIB te lozinka
					
				\end{packed_enum}
			
				\item  \underbar{Roditelj (inicijator) može:}
				
				\begin{packed_enum}
					
					\item pregledavati i mijenjati svoje osobne podatke na svom profilu (mail poslodavca)
					\item pregledavati i mijenjati osobne podatke svoje djece na njihovim profilima (mail škole/vrtića)
					\item pregledavati obavijesti o odobrenom bolovanju od strane liječnika ili poslanoj ispričnici u školu djeteta
					\item učitati nalaz dobiven temeljem usluge u privatnoj ustanovi te za njega zatražiti povratnu informaciju od liječnika ili pedijatra
					\item pregledavati obavijesti o pristiglim nalazima iz laboratorija
					\item pregledavati potvrde o naručivanju na određeni pregled za sebe ili svoju djecu s prikazanom lokacijom pregleda
					\item pregledavati povijest posjeta liječniku i dijagnoze za sebe i svoju djecu
				\end{packed_enum}
				
				\item  \underbar{Liječnik obiteljske medicine (inicijator) može:}
				
				\begin{packed_enum}
					
					\item pregledavati popis pacijenata prijavljenih kod njega (roditelja)
					\item evidentirati pregled pacijenta te događaje na njemu te utvrditi bolest čime se šalje mail poslodavcu
					\item odobriti preporuku za bolovanje za roditelja koju je izdao pedijatar
					\item poslati obavijest roditelju o pristiglom nalazu iz laboratorija
					\item pregledavati nalaze koji su u sustav učitani od strane roditelja te odgovoriti na njih
					\item naručiti pacijenta na specijalistički pregled
				\end{packed_enum}
				
				\item  \underbar{Pedijatar (inicijator) može:}
				
				\begin{packed_enum}
					
					\item pregledavati popis sve neprijavljene djece i prijaviti ih kod sebe
					\item pregledavati popis sve djece prijavljene kod njega
					\item izdati preporuku za bolovanje roditelju čije je dijete bolesno
					\item evidentirati pregled djeteta te događaje na njemu te utvrditi bolest čime se šalje ispričnica u školu
					\item poslati obavijest roditelju o pristiglom nalazu iz laboratorija
					\item pregledavati nalaze koji su u sustav učitani od strane roditelja te odgovoriti na njih
					\item naručiti dijete na specijalistički pregled
					
				\end{packed_enum}
				
				\item  \underbar{Administrator (inicijator) može:}
				
				\begin{packed_enum}
					
					\item vidjeti popis svih registriranih korisnika i njihovih osobnih podataka
					\item brisati korisnike
					\item mijenjati ulogu korisnika (liječnik obiteljske medicine, pedijatar, roditelj)
					\item unijeti registre djece i roditelja (za svaku osobu ime, prezime, OIB i adresu)
					\item registriranom roditelju pridijeliti liječnika obiteljske medicine
					\item djetetu pridijeliti pedijatra
					\item registriranog roditelja povezati s djetetom čiji podaci postoje u sustavu
					
				\end{packed_enum}
				
				\item  \underbar{Baza podataka (sudionik):}
				
				\begin{packed_enum}
					
					\item pohranjuje podatke o svim registriranim korisnicima i njihovim ulogama
					\item pohranjuje podatke o postojećim ustanovama i pregledima koje je moguće obaviti u svakoj
					
				\end{packed_enum}
				
				
			\end{packed_enum}
			
			\eject 
			
			
				
			\subsection{Obrasci uporabe}
				
				\textbf{\textit{dio 1. revizije}}
				
				\subsubsection{Opis obrazaca uporabe}
					\textit{Funkcionalne zahtjeve razraditi u obliku obrazaca uporabe. Svaki obrazac je potrebno razraditi prema donjem predlošku. Ukoliko u nekom koraku može doći do odstupanja, potrebno je to odstupanje opisati i po mogućnosti ponuditi rješenje kojim bi se tijek obrasca vratio na osnovni tijek.}\\
					

					\noindent \underbar{\textbf{UC1 - Registracija}}
					\begin{packed_item}
	
						\item \textbf{Glavni sudionik: }Korisnik
						\item  \textbf{Cilj:} Stvoriti korisnički račun za pristup sustavu
						\item  \textbf{Sudionici:} Baza podataka
						\item  \textbf{Preduvjet:} Administrator je u bazi podataka unio osnovne informacije o korisniku (ime, prezime, OIB)
						\item  \textbf{Opis osnovnog tijeka:}
						
						\item[] \begin{packed_enum}
	
							\item Korisnik na početnoj stranici stisne na opciju za registraciju.
							\item Korisnik unosi svoj OIB te željenu lozinku
							\item Korisnik prima obavijest o uspješnoj registraciji
						\end{packed_enum}
						
						\item  \textbf{Opis mogućih odstupanja:}
						
						\item[] \begin{packed_item}
	
							\item[2.a] Odabir OIB-a za koji je zabilježeno da se osoba već registrirala u sustav
							\item[] \begin{packed_enum}
								
								\item Sustav obavještava korisnika o neuspjeloj registraciji i prikaže mu poruku greške
								
								
							\end{packed_enum}
							
							
						\end{packed_item}
					\end{packed_item}
					
					\noindent \underbar{\textbf{UC2 - Prijava u sustav}}
					\begin{packed_item}
						
						\item \textbf{Glavni sudionik: }Korisnik
						\item  \textbf{Cilj:} Dobiti pristup korisničkom sučelju
						\item  \textbf{Sudionici:} Baza podataka
						\item  \textbf{Preduvjet:} Korisnik je registriran
						\item  \textbf{Opis osnovnog tijeka:}
						
						\item[] \begin{packed_enum}
							
							\item Korisnik na početnoj stranici stisne na opciju za prijavu u sustav.
							\item Korisnik unosi svoj OIB te odgovarajuću lozinku
							\item Korisnik prima obavijest o uspješnoj prijavi i preusmjeren je na stranicu svog profila
						\end{packed_enum}
						
						\item  \textbf{Opis mogućih odstupanja:}
						
						\item[] \begin{packed_item}
							
							\item[2.a] OIB nije registriran u sustav ili lozinka ne odgovara navedenom OIB-u.
							\item[] \begin{packed_enum}
								
								\item Sustav obavještava korisnika o neuspjeloj prijavi i prikaže mu poruku greške.
		
								
							\end{packed_enum}
							
						\end{packed_item}
					\end{packed_item}
					
					\noindent \underbar{\textbf{UC3 - Pregled opisa aplikacije}}
					\begin{packed_item}
						
						\item \textbf{Glavni sudionik: }Korisnik
						\item  \textbf{Cilj:} Pregledati osnovne informacije o aplikaciji.
						\item  \textbf{Sudionici:} Baza podataka
						\item  \textbf{Preduvjet:} -
						\item  \textbf{Opis osnovnog tijeka:}
						
						\item[] \begin{packed_enum}
							
							\item Korisnik otvori početnu stranicu aplikacije.
							\item Na početnoj stranici se prikazuju opis i svrha aplikacije te se nude opcije registracije i prijave u sustav (ako korisnik nije prijavljen).
						\end{packed_enum}
						
					
					\end{packed_item}
					
					\noindent \underbar{\textbf{UC4.1 - Pregled profila roditelja}}
					\begin{packed_item}
						
						\item \textbf{Glavni sudionik: }Roditelj
						\item  \textbf{Cilj:} Pregledati svoj profil i osobne podatke na njemu
						\item  \textbf{Sudionici:} Baza podataka
						\item  \textbf{Preduvjet:} Roditelj je prijavljen
						\item  \textbf{Opis osnovnog tijeka:}
						
						\item[] \begin{packed_enum}
							
							\item Roditelj na početnoj stranici odabere svoj profil označen natpisom "Moj profil". 
							\item Roditelj odabere opciju "Pregled osobnih podataka".
							\item Aplikacija prikazuje osobne podatke roditelja (ime, prezime, OIB, adresa, mail poslodavca, liječnik obiteljske medicine).
						\end{packed_enum}
						
						
					\end{packed_item}
					
					\noindent \underbar{\textbf{UC4.2 - Pregled profila djeteta}}
					\begin{packed_item}
						
						\item \textbf{Glavni sudionik: }Roditelj
						\item  \textbf{Cilj:} Pregledati profil djeteta i osobne podatke na njemu
						\item  \textbf{Sudionici:} Baza podataka
						\item  \textbf{Preduvjet:} Roditelj je prijavljen te je administrator povezao roditeljima s njegovom djecom koja postoje u sustavu.
						\item  \textbf{Opis osnovnog tijeka:}
						
						\item[] \begin{packed_enum}
							
							\item Roditelj na početnoj stranici odabere profil djeteta čiji profil želi pregledati. 
							\item Roditelj odabere opciju "Pregled osobnih podataka".
							\item Aplikacija prikazuje osobne podatke djeteta (ime, prezime, OIB, adresa, mail škole/vrtića, pedijatar)
						\end{packed_enum}
						
						\item  \textbf{Opis mogućih odstupanja:}
						
						\item[] \begin{packed_item}
							
							\item[1.a] Dijete još nije povezano s roditeljem u sustavu.
							\item[] \begin{packed_enum}
								
								\item Roditelj neće imati opciju pregleda profila tog djeteta.
								
								
							\end{packed_enum}
						
							
						\end{packed_item}
					\end{packed_item}
					
					\noindent \underbar{\textbf{UC5.1 - Promjena osobnih podataka roditelja}}
					\begin{packed_item}
						
						\item \textbf{Glavni sudionik: }Roditelj
						\item  \textbf{Cilj:} Ažurirati osobne podatke koje roditelj smije mijenjati (mail poslodavca)
						\item  \textbf{Sudionici:} Baza podataka
						\item  \textbf{Preduvjet:} Roditelj je prijavljen
						\item  \textbf{Opis osnovnog tijeka:}
						
						\item[] \begin{packed_enum}
							
							\item Roditelj na početnoj stranici odabere opciju "Moj profil". 
							\item Roditelj odabere opciju "Pregled osobnih podataka".
							\item Na stranici s podacima roditelj bira opciju za promjenu podataka.
							\item Roditelj sprema podatke.
							\item Baza podataka se ažurira.
						\end{packed_enum}
						
						\item  \textbf{Opis mogućih odstupanja:}
						
						\item[] \begin{packed_item}
							
							\item[2.a] Roditelj promijeni svoje osobne podatke ali ne odabere opciju "Spremi promjene".
							\item[] \begin{packed_enum}
								
								\item Sustav obavještava roditelja da nije spremio podatke kada roditelj pokuša izaći iz prozora.
							\end{packed_enum}
							
						\end{packed_item}
					\end{packed_item}
					
					\noindent \underbar{\textbf{UC5.2 - Promjena osobnih podataka djeteta}}
					\begin{packed_item}
						
						\item \textbf{Glavni sudionik: }Roditelj
						\item  \textbf{Cilj:} Ažurirati osobne podatke o dietetu koje roditelj smije mijenjati (adresa, mail škole/vrtića)
						\item  \textbf{Sudionici:} Baza podataka
						\item  \textbf{Preduvjet:} Roditelj je prijavljen i dijete je povezano s roditeljem u sustavu
						\item  \textbf{Opis osnovnog tijeka:}
						
						\item[] \begin{packed_enum}
							
							\item Roditelj na početnoj stranici bira profil djeteta čije podatke želi promijeniti. 
							\item Roditelj odabere opciju "Pregled osobnih podataka".
							\item Na stranici s podacima djeteta bira opciju za promjenu podataka.
							\item Roditelj sprema podatke.
							\item Baza podataka se ažurira.
						\end{packed_enum}
						
						\item  \textbf{Opis mogućih odstupanja:}
						
						\item[] \begin{packed_item}
							
							\item[2.a] Roditelj promijeni osobne podatke dijeteta ali ne odabere opciju "Spremi promjene".
							\item[] \begin{packed_enum}
								
								\item Sustav obavještava roditelja da nije spremio podatke kada roditelj pokuša izaći iz prozora.
							\end{packed_enum}
							
						\end{packed_item}
					\end{packed_item}
					
					\noindent \underbar{\textbf{UC6.1 - Učitavanje nalaza roditelja u sustav}}
					\begin{packed_item}
						
						\item \textbf{Glavni sudionik: }Roditelj
						\item  \textbf{Cilj:} Učitati postojeći nalaz u sustav i poslati ga liječniku na pregled
						\item  \textbf{Sudionici:} Baza podataka
						\item  \textbf{Preduvjet:} Roditelj je prijavljen
						\item  \textbf{Opis osnovnog tijeka:}
						
						\item[] \begin{packed_enum}
							
							\item Roditelj na početnoj stranici bira opciju "Moj profil".
							\item Roditelj na svojoj stranici bira opciju "Ućitaj nalaz".
							\item Roditelj učita nalaz u sustav i opiše ga ili postavlja pitanje ako to želi.
							\item Roditelj odabere opciju "Pošalji" čime se nalaz šalje liječniku obiteljske medicine.
							\item Baza podataka se ažurira.
						\end{packed_enum}
						
						\item  \textbf{Opis mogućih odstupanja:}
						
						\item[] \begin{packed_item}
							
							\item[2.a] Roditelj učita nalaz ali ne odabere opciju "Pošalji".
							\item[] \begin{packed_enum}
								
								\item Sustav obavještava roditelja da nije poslao podatke kada roditelj pokuša izaći iz prozora.
							\end{packed_enum}
							
							
						\end{packed_item}
					\end{packed_item}
					
						\noindent \underbar{\textbf{UC6.2 - Učitavanje nalaza djetata u sustav}}
					\begin{packed_item}
						
						\item \textbf{Glavni sudionik: }Roditelj
						\item  \textbf{Cilj:} Učitati postojeći nalaz u sustav i poslati ga pedijatru na pregled
						\item  \textbf{Sudionici:} Baza podataka
						\item  \textbf{Preduvjet:} Roditelj je prijavljen
						\item  \textbf{Opis osnovnog tijeka:}
						
						\item[] \begin{packed_enum}
							
							\item Roditelj na početnoj stranici bira profil djeteta čiji nalaz želi učitati u sustav.
							\item Roditelj na djetetovoj stranici bira opciju "Ućitaj nalaz".
							\item Roditelj učita nalaz u sustav i opiše ga ili postavlja pitanje ako to želi.
							\item Roditelj odabere opciju "Pošalji" čime se nalaz šalje pedijatru.
							\item Baza podataka se ažurira.
						\end{packed_enum}
						
						\item  \textbf{Opis mogućih odstupanja:}
						
						\item[] \begin{packed_item}
							
							\item[2.a] Roditelj učita nalaz ali ne odabere opciju "Pošalji".
							\item[] \begin{packed_enum}
								
								\item Sustav obavještava roditelja da nije poslao podatke kada roditelj pokuša izaći iz prozora.
							\end{packed_enum}
							
							
						\end{packed_item}
					\end{packed_item}
					
						\noindent \underbar{\textbf{UC7 - Pregled podataka o naručenom pregledu}}
					\begin{packed_item}
						
						\item \textbf{Glavni sudionik: }Roditelj
						\item  \textbf{Cilj:} Pregledati podatke o naručenom pregledu za sebe ili dijete (vrsta pregleda i lokacije na kojima se može obaviti)
						\item  \textbf{Sudionici:} Baza podataka
						\item  \textbf{Preduvjet:} Roditelj je prijavljen
						\item  \textbf{Opis osnovnog tijeka:}
						
						\item[] \begin{packed_enum}
							
							\item Roditelj na početnoj stranici bira profil osobe za koju želi pregledati naručene preglede (sebe ili jedno dijete).
							\item Roditelj odabere jednu od obavijesti s naslovom "[NARUČEN PREGLED]".
							\item Nakon odabira neke od odgovarajućih obavijesti, roditelj može vidjeti koja je vrsta pregleda te može na prikazanom OpenStreetMap pregledu vidjeti u kojim najbližim zdravstvenim ustanovama (s obzirom na adresu roditelja) se pregled može obaviti.
						\end{packed_enum}
						
						\item  \textbf{Opis mogućih odstupanja:}
						
						\item[] \begin{packed_item}
							
							\item[3.a] Roditelj nije unio svoju adresu u sustav.
							\item[] \begin{packed_enum}
								
								\item Sustav obavještava roditelja da za prikaz mogućih zdravstvenih institucija na mapi roditelj mora ažurirati podatak o svojoj adresi na svom profilu.
							\end{packed_enum}
							
							
						\end{packed_item}
					\end{packed_item}
					
					\noindent \underbar{\textbf{UC8 - Pregled podataka o odobrenom bolovanju}}
					\begin{packed_item}
						
						\item \textbf{Glavni sudionik: }Roditelj
						\item  \textbf{Cilj:} Pregledati podatke o odobrenom bolovanju: razlog bolovanja i trajanje bolovanja
						\item  \textbf{Sudionici:} Baza podataka
						\item  \textbf{Preduvjet:} Roditelj je prijavljen
						\item  \textbf{Opis osnovnog tijeka:}
						
						\item[] \begin{packed_enum}
							
							\item Roditelj na početnoj stranici bira opciju "Moj profil".
							\item Roditelj odabere jednu od obavijesti s naslovom "[BOLOVANJE]".
							\item Nakon odabira neke od odgovarajućih obavijesti, roditelj može vidjeti razlog bolovanja (bolest roditelja ili bolest djeteta), trajanje bolovanja te informacija da je odgovarajući mail poslan poslodavcu.
						\end{packed_enum}
						
						\item  \textbf{Opis mogućih odstupanja:}
						
						\item[] \begin{packed_item}
							
							\item[3.a] Roditelj nije unio mail adresu svog poslodavca.
							\item[] \begin{packed_enum}
								
								\item Unutar obavijesti će biti naznačeno da mail nije poslan poslodavcu jer roditelj u sustav nije unio taj mail.
							\end{packed_enum}
							
							
						\end{packed_item}
					\end{packed_item}
					
					
					\noindent \underbar{\textbf{UC9 - Pregled obavijesti o poslanom mailu vrtiću/školi}}
					\begin{packed_item}
						
						\item \textbf{Glavni sudionik: }Roditelj
						\item  \textbf{Cilj:} Pregledati obavijesti o poslanoj ispričnici vrtiću ili školi
						\item  \textbf{Sudionici:} Baza podataka
						\item  \textbf{Preduvjet:} Roditelj je prijavljen, roditelj je na profilu djeteta unio podatak o mail adresi vrtića ili škole
						\item  \textbf{Opis osnovnog tijeka:}
						
						\item[] \begin{packed_enum}
							
							\item Roditelj na stranici bira profil djeteta za kojeg se žele pregledati poslane ispričnice.
							\item Roditelj odabere jednu od obavijesti s naslovom "[POSLANA ISPRIČNICA]".
							\item Nakon odabira neke od odgovarajućih obavijesti, roditelj može vidjeti na koji mail je poslana ispričnica.
						\end{packed_enum}
						
						
					\end{packed_item}
					
					\noindent \underbar{\textbf{UC10.1 - Pregled obavljenih pregleda i dijagnoza roditelja}}
					\begin{packed_item}
						
						\item \textbf{Glavni sudionik: }Roditelj
						\item  \textbf{Cilj:} Pregledati podatke o prošlim pregledima i dijagnozama
						\item  \textbf{Preduvjet:} Roditelj je prijavljen
						\item  \textbf{Opis osnovnog tijeka:}
						
						\item[] \begin{packed_enum}
							
							\item Roditelj na početnoj stranici bira opciju "Moj profil".
							\item Roditelj odabere jednu od obavijesti s naslovom "[OBAVLJENI PREGLED]".
							\item Nakon odabira neke od odgovarajućih obavijesti, roditelj može vidjeti informacije o obavljenom pregledu te dijagnozi.
						\end{packed_enum}
						
						
					\end{packed_item}
					
					\noindent \underbar{\textbf{UC10.2 - Pregled obavljenih pregleda i dijagnoza djeteta}}
					\begin{packed_item}
						
						\item \textbf{Glavni sudionik: }Roditelj
						\item  \textbf{Cilj:} Pregledati podatke o prošlim pregledima i dijagnozama
						\item  \textbf{Preduvjet:} Roditelj je prijavljen, administrator je povezao roditelja s djetetom
						\item  \textbf{Opis osnovnog tijeka:}
						
						\item[] \begin{packed_enum}
							
							\item Roditelj na početnoj stranici bira profil djeteta za kojeg želi pregledati prošle preglede.
							\item Roditelj odabere jednu od obavijesti s naslovom "[OBAVLJENI PREGLED]".
							\item Nakon odabira neke od odgovarajućih obavijesti, roditelj može vidjeti informacije o obavljenom pregledu te dijagnozi.
						\end{packed_enum}
						
						
					\end{packed_item}
					
					\noindent \underbar{\textbf{UC11 - Pregled popisa djece prijavljene kod nekog pedijatra}}
					\begin{packed_item}
						
						\item \textbf{Glavni sudionik: }Pedijatar
						\item  \textbf{Cilj:} Pregledati popis djece prijavljene kod njega
						\item  \textbf{Preduvjet:} Pedijatar je prijavio djecu kod sebe
						\item  \textbf{Opis osnovnog tijeka:}
						
						\item[] \begin{packed_enum}
							
							\item Pedijatar nakon prijave na početnoj stranici može vidjeti popis djece prijavljene kod njega.
						\end{packed_enum}
						
						
					\end{packed_item}
					
					\noindent \underbar{\textbf{UC12 - Prijava novog djeteta kod pedijatra}}
					\begin{packed_item}
						
						\item \textbf{Glavni sudionik: }Pedijatar
						\item  \textbf{Cilj:} Pregledati popis neprijavljene djece i prijaviti ih kod sebe
						\item  \textbf{Preduvjet:} Administrator je unio podatke o djeci, pedijatar je prijavljen u sustav
						\item  \textbf{Opis osnovnog tijeka:}
						
						\item[] \begin{packed_enum}
							
							\item Pedijatar na početnoj stranici bira opciju "Prijavi novo dijete".
							\item Pedijatar u popisu neprijavljene djece pronalazi dijete koje želi prijaviti kod sebe i odabere opciju "Prijavi".
							\item Dijete se sada nalazi u popisu prijavljene djece na početnoj stranici.
						\end{packed_enum}
						
						
					\end{packed_item}
					
					\noindent \underbar{\textbf{UC13 - Upis podataka o pregledu djeteta obavljenom kod pedijatra i dijagnoza}}
					\begin{packed_item}
						
						\item \textbf{Glavni sudionik: }Pedijatar
						\item  \textbf{Cilj:} Upisati podatak o obavljenom pregledu djeteta
						\item  \textbf{Preduvjet:} Pedijatar je prijavljen u sustav, dijete je prijavljeno kod pedijatra
						\item  \textbf{Opis osnovnog tijeka:}
						
						\item[] \begin{packed_enum}
							
							\item Pedijatar na početnoj stranici iz popisa djece prijavljene kod njega bira dijete čiji pregled želi unijeti.
							\item Pedijatar bira opciju "Dijagnoza".
							\item Pedijatar opiše pregled i dijagnozu te opcionalno može odabrati šalje li se ispričnica i preporuka za bolovanje roditelja.
							\item Pedijatar bira opciju "Pošalji".
						\end{packed_enum}
						
						\item  \textbf{Opis mogućih odstupanja:}
						
						\item[] \begin{packed_item}
							
							\item[4.a] Pedijatar nije odabrao opciju "Pošalji".
							\item[] \begin{packed_enum}
								
								\item Aplikacija pedijatra obaviještava o tome da nije odabrao opciju "Pošalji" kada pedijatar pokuša zatvoriti stranicu.
							\end{packed_enum}
							
							
						\end{packed_item}
						
						
					\end{packed_item}
					
					
					\noindent \underbar{\textbf{UC14 - Izdavanje preporuke za bolovanje za roditelja bolesnog djeteta}}
					\begin{packed_item}
						
						\item \textbf{Glavni sudionik: }Pedijatar
						\item  \textbf{Cilj:} Izdati preporuku za bolovanje roditelju bolesnog djeteta
						\item  \textbf{Preduvjet:} Pedijatar je prijavljen u sustav, dijete je prijavljeno kod pedijatra
						\item  \textbf{Opis osnovnog tijeka:}
						
						\item[] \begin{packed_enum}
							
							\item Pedijatar na početnoj stranici iz popisa djece prijavljene kod njega bira dijete čijem roditelju želi izdati preporuku za bolovanje.
							\item Pedijatar bira opciju "Dijagnoza".
							\item Pedijatar opiše razlog izdavanja preporuke te bira opciju "Preporuka za bolovanje roditelja" i opciju "Ispričnica školi/vrtiću".
							\item Pedijatar bira opciju "Pošalji".
						\end{packed_enum}
						
						\item  \textbf{Opis mogućih odstupanja:}
						
						\item[] \begin{packed_item}
							
							\item[4.a] Pedijatar nije odabrao opciju "Pošalji".
							\item[] \begin{packed_enum}
								
								\item Aplikacija pedijatra obaviještava o tome da nije odabrao opciju "Pošalji" kada pedijatar pokuša zatvoriti stranicu.
							\end{packed_enum}
							
							
						\end{packed_item}
						
						
					\end{packed_item}
					
					\noindent \underbar{\textbf{UC15 - Slanje nalaza iz laboratorija djeteta}}
					\begin{packed_item}
						
						\item \textbf{Glavni sudionik: }Pedijatar
						\item  \textbf{Cilj:} Roditelju djeteta poslati laboratorijski nalaz djeteta
						\item  \textbf{Preduvjet:} Pedijatar je prijavljen u sustav, dijete je prijavljeno kod pedijatra
						\item  \textbf{Opis osnovnog tijeka:}
						
						\item[] \begin{packed_enum}
							
							\item Pedijatar na početnoj stranici iz popisa djece prijavljene kod njega bira dijete čiji pregled želi unijeti.
							\item Pedijatar bira opciju "Nalaz iz laboratorija".
							\item Pedijatar opiše nalaz te može priložiti dokument biranjem opcije "Prilog" te opcionalno može odabrati šalje li se ispričnica i preporuka za bolovanje roditelja.
							\item Pedijatar bira opciju "Pošalji".
						\end{packed_enum}
						
						\item  \textbf{Opis mogućih odstupanja:}
						
						\item[] \begin{packed_item}
							
							\item[4.a] Pedijatar nije odabrao opciju "Pošalji".
							\item[] \begin{packed_enum}
								
								\item Aplikacija pedijatra obaviještava o tome da nije odabrao opciju "Pošalji" kada pedijatar pokuša zatvoriti stranicu.
							\end{packed_enum}
							
							
						\end{packed_item}
						
						
					\end{packed_item}
					
					\noindent \underbar{\textbf{UC16 - Pregled učitanih nalaza djeteta od strane roditelja}}
					\begin{packed_item}
						
						\item \textbf{Glavni sudionik: }Pedijatar
						\item  \textbf{Cilj:} Pregledati nalaze djeteta koje su roditelji učitali u sustav
						\item  \textbf{Preduvjet:} Pedijatar je prijavljen u sustav, dijete je prijavljeno kod pedijatra, pedijatar je primio obavijest o učitanom nalazu
						\item  \textbf{Opis osnovnog tijeka:}
						
						\item[] \begin{packed_enum}
							
							\item Pedijatar na početnoj stranici iz popisa djece prijavljene kod njega bira dijete čije nalaze želi vidjeti.
							\item Pedijatar bira jednu od obavijesti s naslovom [UČITAN NALAZ].
							\item Pedijatar može pregledati nalaz te dodatna pitanja ili informacije koje je roditelj priložio.
						\end{packed_enum}
						
						
					\end{packed_item}
					
					\noindent \underbar{\textbf{UC17 - Davanje povratne informacije o učitanom nalazu djeteta}}
					\begin{packed_item}
						
						\item \textbf{Glavni sudionik: }Pedijatar
						\item  \textbf{Cilj:} Odgovoriti na nalaz koji je roditelj učitao u sustav
						\item  \textbf{Preduvjet:} Pedijatar je prijavljen u sustav, dijete je prijavljeno kod pedijatra, pedijatar je primio obavijest o učitanom nalazu
						\item  \textbf{Opis osnovnog tijeka:}
						
						\item[] \begin{packed_enum}
							
							\item Pedijatar na početnoj stranici iz popisa djece prijavljene kod njega bira dijete čije nalaze želi vidjeti.
							\item Pedijatar bira jednu od obavijesti s naslovom [UČITAN NALAZ].
							\item Pedijatar odabere opciju "Odgovori" i sastavlja svoj odgovor.
							\item Pedijatar odabere opciju "Pošalji".
						\end{packed_enum}
						
						\item  \textbf{Opis mogućih odstupanja:}
						
						\item[] \begin{packed_item}
							
							\item[3.a] Pedijatar nije odabrao opciju "Pošalji".
							\item[] \begin{packed_enum}
								
								\item Aplikacija pedijatra obaviještava o tome da nije odabrao opciju "Pošalji" kada pedijatar pokuša zatvoriti stranicu.
							\end{packed_enum}
							
							
						\end{packed_item}
						
						
					\end{packed_item}
					
					\noindent \underbar{\textbf{UC18 - Naručivanje djeteta na specijalistički pregled}}
					\begin{packed_item}
						
						\item \textbf{Glavni sudionik: }Pedijatar
						\item  \textbf{Cilj:} Naručiti dijete na specijalistički pregled
						\item  \textbf{Preduvjet:} Pedijatar je prijavljen u sustav, dijete je prijavljeno kod pedijatra
						\item  \textbf{Opis osnovnog tijeka:}
						
						\item[] \begin{packed_enum}
							
							\item Pedijatar na početnoj stranici iz popisa djece prijavljene kod njega bira dijete koje želi naručiti na specijalistički pregled.
							\item Pedijatar odabere opciju "Naručivanje specijalističkog pregleda.
							\item Pedijatar odabere vrstu pregleda te dodaje napomenu ako to želi.
							\item Pedijatar odabere opciju "Pošalji".
						\end{packed_enum}
						
						\item  \textbf{Opis mogućih odstupanja:}
						
						\item[] \begin{packed_item}
							
							\item[4.a] Pedijatar nije odabrao opciju "Pošalji".
							\item[] \begin{packed_enum}
								
								\item Aplikacija pedijatra obaviještava o tome da nije odabrao opciju "Pošalji" kada pedijatar pokuša zatvoriti stranicu.
							\end{packed_enum}
							
							
						\end{packed_item}
						
						
					\end{packed_item}
						
				
					
				\subsubsection{Dijagrami obrazaca uporabe}
					
					\textit{Prikazati odnos aktora i obrazaca uporabe odgovarajućim UML dijagramom. Nije nužno nacrtati sve na jednom dijagramu. Modelirati po razinama apstrakcije i skupovima srodnih funkcionalnosti.}
				\eject		
				
			\subsection{Sekvencijski dijagrami}
				
				\textbf{\textit{dio 1. revizije}}\\
				
				\textit{Nacrtati sekvencijske dijagrame koji modeliraju najvažnije dijelove sustava (max. 4 dijagrama). Ukoliko postoji nedoumica oko odabira, razjasniti s asistentom. Uz svaki dijagram napisati detaljni opis dijagrama.}
				\eject
	
		\section{Ostali zahtjevi}
		
			\textbf{\textit{dio 1. revizije}}\\
		 
			 \textit{Nefunkcionalni zahtjevi i zahtjevi domene primjene dopunjuju funkcionalne zahtjeve. Oni opisuju \textbf{kako se sustav treba ponašati} i koja \textbf{ograničenja} treba poštivati (performanse, korisničko iskustvo, pouzdanost, standardi kvalitete, sigurnost...). Primjeri takvih zahtjeva u Vašem projektu mogu biti: podržani jezici korisničkog sučelja, vrijeme odziva, najveći mogući podržani broj korisnika, podržane web/mobilne platforme, razina zaštite (protokoli komunikacije, kriptiranje...)... Svaki takav zahtjev potrebno je navesti u jednoj ili dvije rečenice.}
			 
			 
			 
	