\chapter{Zaključak i budući rad}
		
		 
		 Cilj rada na ovom projektu bio je napraviti aplikaciju koja omogućava roditeljima s često bolesnom djecom i liječnicima te pedijatrima lakše dogovaranje oko pregleda, slanje nalaza, slanje ispričnica te odobravanje bolovanja. Tijekom rada na projektu najveći je izazov bio što nijedan član tima nije bio upoznat s načinom izrade aplikacija koristeći Spring Boot i React. Članovi su puno vremena uložili u samo upoznavanje s tehnologijama te bi ovaj projekt bio puno lakši da su ta znanja već bila stečena. 
		 
		 Rad na projektu trajao je 11 tjedana. U prvih 5 tjedana članovi su stekli znanja o tome kako organizirati funkcionalne i ostale zahtjeve sustava te iz njih napisati obrasce uporabe. Naučili su pomoću njih izraditi sekvencijski dijagram te dijagram obrazaca uporabe. Također su se upoznali s arhitekturom mobilne aplikacije te izradom baze podataka.
		 
		 U idućih 6 tjedana članovi su uglavnom radili na implementaciji projekta. Upoznali su se sa strukturom koda u Springu te načinu funkcioniranja React stranica. Veliki je dio izazova bio i sinkronizacija backenda i frontenda jer se prilikom njihovog prvotnog spajanja uočilo puno greški. Osim toga, članovi su se upoznali i s ostatkom UML dijagrama: dijagram stanja, aktivnosti, komponenti te razmještaja.
		 
		 Sudjelovanje na projektu bilo je vrlo korisno iskustvo. Članovi su osim stečenih znanja o navedenih tehnologija imali priliku iskusiti timski rad u kojem su organizacija i međusobno pomaganje ključni za uspjeh. Ovo je također mnogim članovima bio prvi doticaj s ozbiljnijim projektom u kojem postoje strogi vremenski rokovi. Svi su članovi zadovoljni sa stečenim znanjem i iskustvom.
		
		\eject 